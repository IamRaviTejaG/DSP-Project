\documentclass[a4paper, 12pt]{article}

\usepackage{titling}
\usepackage[left=1cm, right=1cm, top=0cm, bottom=0cm]{geometry}
\pagenumbering{gobble}

%\newcommand{\subtitle}[1]{% 
%	\posttitle{%
%		\par \end{center}
%		\begin{center}\small#1\end{center}
%		\vskip0.5em}%
%}

\begin{document}

\title{\huge Huffman Encoding and Decoding using
MATLAB\\\rule{15cm}{0.01cm}}
%\rule{10cm}{0.1cm}
\author{\Large Ravi Teja Gannavarapu}
\date{\large B216023\\\rule{15cm}{0.01cm}}
\maketitle

\begin{center}
\textbf{\Large Abstract}
\end{center}

\Large Data, when being transmitted over large distances and via different channels, requires to be sent securely. Encoding the
information before transmission is necessary to ensure data security and efficient delivery of the information. \textbf{Huffman algorithm} is a popular encoding method used in communication systems. Huffman encoding and decoding algorithm is used in compressing data with variable-length codes. The shortest codes are assigned to the most frequent characters and the longest codes are assigned to infrequent characters. Huffman coding is an entropy encoding algorithm used for \textbf{lossless} data compression. \textbf{Entropy} is a measure of the unpredictability of an information stream. Maximum entropy occurs when a stream of data has totally unpredictable bits. A perfectly consistent stream of bits (all zeroes or all ones) is totally predictable, and has no entropy.
\end{document}